\documentclass{ctexart}

% Language setting
% Replace `english' with e.g. `spanish' to change the document language
\usepackage[english]{babel}

% Set page size and margins
% Replace `letterpaper' with `a4paper' for UK/EU standard size
\usepackage[letterpaper,top=2cm,bottom=2cm,left=3cm,right=3cm,marginparwidth=1.75cm]{geometry}

% Useful packages
\usepackage{amsmath}
\usepackage{graphicx}
\usepackage[colorlinks=true, allcolors=blue]{hyperref}

\title{中国科学技术大学交流生生存指北}
\author{ }
\date{ }

\begin{document}
\maketitle

\section{前言}
同学,当你看到这份生存指北的时候,首先恭喜你,你已经经过学校的重重选拔,即将作为一名交流生踏入中国科学技术大学的校园.相信你现在的内心一定是充满激动和喜悦的,一定是憧憬着在中国科学技术大学的交流生活的。这份生存指北将为你未来的生活提供参考.
\par 中国科学技术大学始建于1958年,后因文化大革命南迁至安徽合肥,秉持着{\bf{\textcolor{red}{红专并进,理实交融}}}的校训,迎着永恒的东风,已经走过66年(截止到2024年),学校设有32个学院,本科专业41个,含8个科教融合学院,是一所传统的理科强校

\end{document}
